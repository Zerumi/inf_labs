\documentclass[9pt, a4paper]{article}
\usepackage[utf8]{inputenc}
\usepackage[russian]{babel}
\usepackage{multicol}
\usepackage{fancyhdr}
\usepackage{graphicx}
\usepackage{booktabs}
\usepackage{cuted}
\usepackage[left=25, right=25, top=1in, bottom=1in]{geometry}
\usepackage{tabularx}
\usepackage{flowfram}
\usepackage{lipsum}
\usepackage{wrapfig}

\pagestyle{fancy}
\fancyhead{}
\fancyhead[L]{50}
\fancyhead[C]{\textbf{КВАНТ · 1995/№1}}
\fancyhead[R]{}

\begin{document}
\pagenumbering{gobble}
\begin{footnotesize}
\setlength{\parindent}{8pt} 
\begin{multicols}{3}

Тогда результаты, получекные в главах II и III, можно записать в виде формулы

$
P(n) \leq I(n) \leq Q(n)
$

Напомним, что

$
P(n)=[n(n-1)]+n-3,
$

$
Q(n)=2l[n]+n-3,
$

где $\{n]=k$ при $2^k \geq n>2^{k-1}$

Можно показать, что $Q(n)$ равно либо $P(n)$, либо $P(n)+1$. В первом случае $I(n)=P(n)=Q(n)$. Во втором случае для $I(n)$ имеются две возможности: либо $I(n)=P(n)$, либо $I(n)=Q(n)$. Обе возможности могут осуществляться, как видно из приведенной таблицы.
\begin{wraptable}[4]{i}{2\linewidth}
\resizebox{2\columnwidth}{!}{%
\begin{tabular}{|l|l|l|l|l|l|l|l|l|l|l|l|l|l|l|l|l|l|l|}
\hline$n$ & 3 & 4 & 5 & 6 & 7 & 8 & 9 & 10 & 11 & 12 & 13 & 14 & 15 & 16 & 17 & 18 & 19 & 20 \\
\hline$P(n)$ & 3 & 5 & 7 & 8 & 10 & 11 & 13 & 14 & 15 & 17 & 18 & 19 & 20 & 21 & 23 & 24 & 25 & 26 \\
\hline$I(n)$ & 3 & 5 & 7 & 9 & 10 & 11 & 13 & 14 & 16 & 17 & 18 & 19 & 20 & 21 & 23 & 24 & 25 & 26 \\
\hline$Q(n)$ & 4 & 5 & 8 & 9 & 10 & 11 & 14 & 15 & 16 & 17 & 18 & 19 & 20 & 21 & 24 & 25 & 26 & 27 \\
\hline
\end{tabular}%
}
\end{wraptable}
\break
\break
\break
\break

Как именно ведет себя $I(n)$, мы расскажем подробнее в другой раз. А пока попытайтесь разобраться в этом самостоятельно. Попробуйте решить также следующие задачи.

1. Проверьте прилагаемую выше таблину.
Указание. Наметим путь рениеиия этой задачи дли $n=20$. В зтом случае $P[n]=P[20]=26$.
Разделим 20 шахматистов на две группы: в первой группе 16 человек, во второй - четыре.

\columnbreak

В первой группе определим стандартным способом (см. §1 и §2 главы III) 1-го призера $A$ и 2-го иризера $B$. На это уйдет $15+3=18$ партий. При этом $B$ выйграет не больше чем у четырех человек (см. §3).

Во второй групnе определим сильнейшего $\mathrm{C}$ по олимпийской системе (3 партии). При этом $C$ выиграет у двоих.

Слелущую (22-ю) партию проведем между $B$ и $C$. Если победит $B$, то ясно, что $A$ чемпион, $B$ - второй призер, а на третье место претендуют пятеро, проигравших B. За оставшиеся 4 партии можно найти среди них третьего призера.

Пусть, наоборот, победит $C$. Тогда вознккнет следующая ситуация (рис. 1): на призовые места претендуют шахматисты $A$, $B$, $C$, $D$, $E$ и $F$ (стрелки ведут от победителей к побежденным. Проведем 23-ко партию между $D$ и $E$ : пусть в этой партии победит $E$ (случай, когда победит $D$, проще и рассматривается аналогично).

\hfill 
\\[25pt] 
\break

24-ю партию проведем между $A$ и $E$. Если выйграет $E$, то $C$ - чемпион, $E$ - второй призер, а на третье место претендуют $A$, $D$ и $F$. За оставшиеся 2 партии найдем среди них третьего призера.

Если 24 -ю партию выйграет $A$, то (см. рис.2) на первые два места претендуют $A$ и $C$, на третье - $B$ и E. Проведя две партии (между $A$ и С и между $B$ и $E$ ), мы опрелелим 1-го, 2-го и 3-го призеров. Итак, $I(20)=26=P(20)$.

\columnbreak
\begin{center}
    \includegraphics[width=0.4\linewidth]{images/graph1.png}
    \includegraphics[width=0.4\linewidth]{images/graph2.png}
\end{center}

Только что приведенные правила резко отличаются от стандартных правил из главы III. По стандартным правилам шахматисты, проигравшие хоть одну пртию, не участвуют в следующих играх до тех пор, пока не определится чемпинон. Только отказавшись от этого, нам удалось определить 1-го, 2-го и 3-го призеров среди 20 шахматистов за 26 (а не за 27) партий.

2. Докажите, что 1-го и 2-го призеров среди $n$ шахматистов наверняка можно определить за $l[n] + n$ - 2 партий и может не удастся определить за меньшее число партий.

3. Пусть мы хотим определить среди n шахматистов k сильнейших (1-го, 2-го, ... $k$-го призеров). Докажите, что
а) это наверняка можно сделать за (k-1)l[n]+n-k партий;
б) если

$
R < l[n(n-1)...(n-k+2)]+n-k
$

то результаты партий могут оказаться такими, что это не удастся сделать за $R$ партий.

Указание. В задаче б) мы рекомендуем рассуждать так же, как и в главе II, рассматривая "команды" $A_1, A_2,..., A_k-1$, состоящие из k-1 участника. Общее число таких команд равно $n(n-1)...(n-k+2)$.
\end{multicols}

\begin{center}
\large{\textbf{ИНФОРМАЦИЯ}}
\end{center}
\hrule
\begin{center}
\large{\textbf{XXII ЛЕТНЯЯ ФИЗИКО-МАТЕМАТИЧЕСКАЯ ШКОЛА ВО ВЛАДИВОСТОКЕ}}
\end{center}

\begin{multicols}{3}

(Начало см. на с.45)

Как видим, школьникам предлагалось взглянуть на привычные, казалось бы, законы и теоремы под непривычным, не вполне школьным, углом зрения, порешать красивые задачи, в том числе - участвуя в олимпиадах. Победители олимпиад (их было две, одна - чисто математическая, другая - смешанная, физико-математическая) были награждены специальными призами и грамотами журнала "Квант". Кроме того, памятную грамоту журнала "Квант" получил каждый участник Школы. Отметим еще, что все слушатели были приняты в Заочную физико-техническую школу МФТИ.

Если вспомнить, что кроме уроков и олимпиад были еще занятия в спортзале, соревнования, экскурсии, а ведь летом надо еще и отдыхать, и купиться в

\columnbreak

море - становится понятно, что жизнь у ребят была весьма насыщенной. Организацией работы и жизни Школы ведала завуч хабаровского лицея №2 Н. Е. Довбило, а под ее руководством работала группа воспитателей-студентов. Эти молодые люди совсем недавно сами были рядовыми ее участниками, причем некоторые - неоднократно. Именно от них будущие слушатели узнавали о Школе "из первых рук", в частности о том, что практически все прошедшие Школу становятся студентами. Лидером и координатором этой группы студентов был старший воспитатель Евгений Шлямов - студент 5 курса Благовещенского политехнического института. Успешному проведению Школы немало способствовало заботливое отношение директора Приморского краевого ИУУ И. А. Яколева.

\columnbreak

Разумеется, успех столь масштабного предприятия мог состояться только при мощной поддержке организаторов Школы. Это - Институт автоматики и процессов управления ДВО РАН. Российский фонд фундаментальных исследований, Московский физико-технический институт, Инновационный фонд Министерства образования РФ, Управление образования Приморского края, а также Международная Соросовская программа образования в области точных наук. Все участники и преподаватели Школы выражают им самую искреннюю благодарность.

В 1995 году во Владивостоке состоится очередная XXIII Летняя школа. Заявки об участие в ней просим присылать в редакцию с пометкой "Приморская летняя школа".

Авторы: А.Егоров, А.Черноуцан

\end{multicols}
\end{footnotesize}
\end{document}